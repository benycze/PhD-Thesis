\widowpenalty=9999
\chapter{Conclusion}
\label{chap:conclusion}
% Summarize each chapter in one paragraph, point out the most important findings.
The aim of my \thesis{} is to provide the process of mapping from abstract language to the architecture of network device which is 
suitable for automatic generation and capable to hit processing speed of 100\,Gbps.

We propose the Parser-Deparser model. Our model of network device is based on the standard P4 model and it consists of three modules. 
The first module, parser, is used to break the incoming network data into individual header fields
(i.e., it has the same functionality like parser in P4 model).
The output of this module is a set of structured extracted fields, metadata and valid bits. All these inputs are passed
to the processing block which does not strictly define the structure of processing part 
(compared to the standard P4 model where the structure of processing part is given). 
Therefore, we are able to connect strong ideas of P4 model and rapid implementation of novel network functionality 
(e.g., pattern matching, new architectures of network devices, novel classification and so on). 
Finally, the last module, deparser, is used for construction of network packet back from protocol headers and valid bits.
This approach is general enough for variety of network applications. 
The Parser-Deparser model is capable to accommodate the P4 device model 
because the processing part allows mapping of Ingress Match+Action, Queues and/or Buffers and Egress Match+Action 
to predefined architecture.

We propose the architecture of our network pipeline which is suitable for automatic generation and capable to hit high-speeds. 
It is an example of the trade-off between complexity of high-speed network hardware blocks and transformation process. 
The proposed architecture contains several building blocks which are connected via predefined interfaces into deep high-speed pipeline.
We introduced the mapping from higher level (P4 language) to the architecture of parser, deparser, Match+Action tables,
Match+Action routers and Match+Action groups. The description on higher levels allows the user to be focused on implementation of 
network application without deep knowledge of underlying architecture. This approach was demonstrated on two examples --- 
(1)~transformation of P4 description to VHDL and (2)~extension of available action set with usage of current HLS tool where the
description of user action was provided in C/C++ language.
Our results shows that both approaches are capable to produce synthesizable code of a network device 
which is able to hit processing speed of 100\,Gbps.

The transformation from the P4 to VHDL was demonstrated on three use cases --- IPv4 Filter, IPv4+IPv6 Filter and Full Filter.
These use cases show the flexibility and easy extensibility of our architecture with new protocols and actions. The behavior of each use case
was described in P4 language and translated using our tool. We were able to reach the throughput in range from 77.6 to 100\,Gbps
and the frequency of 225.85\,MHz on Xilinx Virtex 7 XCVH580T FPGA. 

The transformation from C/C++ to VHDL was demonstrated on the extension of processing part with user defined actions.
The P4 language defines a fixed set of primitive actions. 
Unfortunately, this limitation is not suitable for future development because we typically need to extend the current action set with new actions 
(e.g., computation of advanced statistics, and so on). The user defined actions can be described in higher language which
is beneficial for fast development. Moreover, the source code of new action can by provided by administrator, mathematician or security expert.
The results of this research were verified in SDM \cite{KekelyPusKorenekSDM} project which is actively used for monitoring and protection 
of computer networks.

\section{Contributions of the Thesis}

The contributions of my \thesis{} are summarized in the following list:
\begin{enumerate}
    \item The modular architecture of network device which is suitable for generation of high-speed packet processing devices 
    from the abstract description.
    \item The structure and transformation process for individual parts of generated high-speed network device: 
    parser, Match+Action router, Match+Action table and deparser.
    \item The architecture of parallel Action Engine which is capable to process traffic at speed of 100\,Gbps. 
    The architecture of this engine was verified in SDM \cite{KekelyPusKorenekSDM} project.
    It is also capable to accommodate High-Level Synthesized engines which are described in C/C++ language.
    Results of this research are used in \cite{KekelyPusKorenekSdmIeeeTransactionComputerNetworks}.
    \item The tool, which generates the proposed architecture of network device from the abstract description in P4 language.
    \item Experimental results for parser, deparser and Match+Action processing pipeline. 
    Finally, the text provides three use cases which were tested 
    against Spirent's Hypermetrics MX-100G-F1 module (SPT-N11U chassis) at speed of 100\,Gbps. 
    Each use case was described in P4 language, translated to VHDL and tested in real hardware environment.
    The provided use cases demonstrate the flexibility and easy extensibility with support of new protocols and actions.
\end{enumerate}

\section{Future Work}
The author of the \thesis{} suggests to explore following:
\begin{itemize}
    \item Provide a general architecture of \textit{Check} engine.
    \item Investigate further optimizations of generated structure 
    (e.g., automatic detection of similar protocols which leads to optimization of PGR's structure).
    \item Provide an effective architecture of deparser block, capable to construct packets at speeds above 100\,Gbps.
    \item Provide architecture of more complex \textit{Search Engine} in Match+Action table.
    \item Extend the Match+Action table with advanced stateful processing (inspired by OpenState \cite{OpenStateSpec}).
    \item Provide a study of this technology for speeds beyond 100\,Gbps (e.g., for upcoming 400\,Gbps Ethernet standard).
    %\item Provide similar technology of processing pipeline for speeds up to 400\,Gbps. 
\end{itemize}

% % % % % % % % % % % % % % % % % % % % % % % % % % % % % % % % % % % % % % % % % % % % % % % % % % % % % % % % % % % %
% % % % Originalni text 
% % % % % % % % % % % % % % % % % % % % % % % % % % % % % % % % % % % % % % % % % % % % % % % % % % % % % % % % % % % %
%Chapter \ref{chap:stateOfTheArt} discusses the current approaches in the area of packet parsing, classification and data processing. 
%The chapter also introduces modern languages and abstractions of packet processing devices. We choose the P4 language as an input
%for our mapping process because this language provides good abstraction, is developed as open standard and is
%considered to be device and platform independent.
%
%Chapter \ref{chap:architectureOfNetworkDevice} introduces possible target architectures for mapping of P4 program. 
%Namely, the chapter introduces details about software implementations capable 
%to process 10\,Gbps, Network Processing Units (NPUs) capable to process traffic at speed of 40\,Gbps, and Field Programmable Gate Array (FPGA)
%which is capable to process data at speed of 100\,Gbps.
%We select the FPGA as our target platform because it connects parallel nature of hardware with flexibility of software. 
%The chapter also introduces the idea of mapping to the architecture of network device which is based on Parser-Deparser model.
%The proposed architecture consists of several building blocks which are connected to processing pipeline where some 
%parts of these blocks are generated, some configured and some instantiated.
%
%Chapter \ref{chap:parserDeparserArchitecture} introduces detailed architecture and transformation process from P4 language to parser and deparser 
%blocks. 
%%Both blocks are carefully designed to meet a trade-off between complexity of hardware and generator. 
%Parser consists of pipeline modules and Protocol Analyzers which are connected via common interface. 
%The same architecture is used in the case of deparser.
%This approach is beneficial during generation of mentioned blocks because the structure of generated hardware is homogenous. 
%The text provides the algorithms for transformation from P4 to the structure of parser and deparser.
%We also compared generated to hand-written parsers. The cost for automatic generation is roughly doubled latency and chip area. 
%Finally, we connected parser and deparser together and perform throughput tests in real hardware platform. 
%Results show that generated blocks are capable to hit speed of 100\,Gbps, except some packet lengths when the inefficiency in deparser shows up. 
%The text also proposes required changes to meet the full throughput of 100\,Gbps.
%
%Chapter \ref{chap:matchActionProcessing} introduces results of our research regarding the Match+Action processing pipeline with further
%details about transformation algorithm and generated architecture. We introduced mapping from P4 language to 
%Match+Action groups, Match+Action tables and Match+Action routers which are the main building blocks of our processing pipeline. 
%All mentioned modules use the common communication interface which is beneficial during automatic generation.
%The text also provides details about usage of current HLS tool which can be used for easy extension of available actions in Action Engine.
%That is, we can provide abstract definition of user-defined engines in C/C++ and translate this code to HDL language. 
%After that, the generated code is connected into existing infrastructure which is capable to process data at speed of 100\,Gbps. 
%
%Finally, Chapter \ref{chap:use-case-study} provides results for three use cases where each use case was generated from P4 description.
%We provided results of required resources, throughput and number of generated lines, together with required time for compilation.
%Each use case meet throughput in range from 77.6 to 100\,Gbps at frequency of 225.8\,MHz on Xilinx Virtex-7 XCVH580T FPGA.
