\chapter{Abstract}
%\chapter*{Abstract and Contributions}
%\addcontentsline{toc}{chapter}{\protect\numberline{}{Abstract and contributions}}

OpenFlow, as the most popular embodiment of Software-Defined Networking, provides a way
to network dataplane configuration at runtime. The OpenFlow specification strictly defines a set of supported protocols 
and actions for further processing of incoming traffic (i.e., switches are still mostly fixed). 
However, modern requirements on networking hardware have a dynamic
character and administrators of high-end networks want to react to new protocols, security threats,
novel approaches in traffic engineering, and so on. This isn't feasible with static network hardware and
leads to the need to replace the hardware more often than desired.

The aim of my \thesis{} is to provide the process of mapping from abstract language to the architecture of network device which is 
suitable for automatic generation and capable to hit processing speed of 100\,Gbps in single FPGA. 
The architecture of network device is based on predefined interfaces and modules which are connected to a high-speed processing pipeline.
The text provides details of transformation process to individual blocks of network 
device: parser, deparser, Match+Action table, Match+Action router and Match+Action group. 
The text also demonstrates the usage of High-Level Synthesis tools to provide a custom 
processing engine which is capable
to hit speed of 100\,Gbps. Finally, the text provides three use cases for demonstration of flexibility and easy extensibility
with new protocols and actions. Each use case was described in P4 language, translated to VHDL and tested in real hardware environment. 
The results show that generated devices are capable to meet throughput in range from 77.6 to 100\,Gbps.

%\bigskip
%\noindent The contributions of my \thesis{} are summarized in the following list:
%\begin{enumerate}
%    \item Introduction of modular architecture which is suitable for automatic generation of high-speed packet processing devices 
%    from abstract description.
%    \item Introduction of structure and mapping process for individual parts of generated packet processing device: 
%    Match+Action router, Match+Action table, parser and deparser. 
%    \item Tool, which generates the proposed architecture of packet processing device from abstract description in P4 language.
%    \item Experimental results of generated packet processing devices, including tests in real hardware platform.
%\end{enumerate}

\bigskip

\noindent\textbf{Keywords:}\\
\indent FPGA, High-Level Synthesis, Transformation, P4, SDN, High-Speed Computer Networks, 100\,Gbps, VHSIC Hardware Description Language.

%\vfill
%If you present in this thesis work achieved jointly with some other people except for your supervisor or co-supervisor, then this is a place to put their statements clarifying your qualitative and quantitative participation.
%For example,

%\noindent As a collaborator of {\FirstandFamilyName} and a co-author of his papers, I agree with {\FirstandFamilyName}'s authorship of the research results, as stated in this dissertation thesis.

%\bigskip\bigskip
%\begin{flushright}
%    \begin{tabular}{p{6cm}}
%           \dotfill\\
%           {\em Name Surname}
%    \end{tabular}
%\end{flushright}
