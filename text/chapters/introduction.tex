\chapter{Introduction}
% \enlargethispage*{5pt}
\label{chap:introduction}

\section{Motivation}
%Brief description of the topic of the \thesis. A complete explanation of the topic shall be described within chapter 
%\ref{chap.stateoftheart} at page \pageref{chap.stateoftheart}.

% Tady popisu to, ze se pocitacove site rychle rozviji a je nutne na to reagovat. Proto se nehodi fixni hardware a software
% a hodne se pouziva konfigurovatelny HW zalozeny na FPGA. Nanestesti jsou HDL jazyky hodne slozite a tak je nemuze
% pouzivat odbornik na bezpecnost, matematik nebo sitovy admin.

OpenFlow \cite{openflow}, as the most popular embodiment of Software-Defined Networking (SDN), provides a way
to network dataplane configuration at runtime. The OpenFlow specification strictly defines a set of supported protocols 
and actions for further processing of incoming traffic (i.e., switches are still mostly fixed). 
However, modern requirements on networking hardware have a dynamic
character and administrators of high-end networks want to react to new protocols, security threats,
novel approaches in traffic engineering, and so on. 
This isn't feasible with static network hardware and leads to the need to replace the hardware more often than desired.

The fixed behavior of network devices isn't also suitable for further scaling of data centers because such technological solutions 
need to react to actual demands on network infrastructure. This need is an impulse for developers and manufactures of networking
hardware because they are motivated to provide highly reconfigurable platforms. The example of such solution is not only the OpenFlow switch
but also the Facebook's 6-pack \cite{FacebookSixPack} or Wedge \cite{FacebookWedge}. Both solutions are equipped with an ASIC chip
from Broadcom (for processing of network data at high-speed) and control software which is used as a substrate for implementation
of control logic. Such solution can be easily reprogrammed to serve as a router, switch, or network filter.

However, such devices cannot be later extended with support of novel classification, parsing, or packet processing functionality
at high-speed (i.e., they are highly reconfigurable but still fixed). 
Due to this, developers and manufactures of network hardware are motivated to equip network devices with reprogrammable chips like FPGAs. 
This platform connects the flexibility of software with parallel nature of hardware into one
compact package. 
There are also available hardware platforms \cite{combo-100g} for development of network applications, including frameworks for rapid 
prototyping \cite{NetCOPEWeb}.

The behavior of implemented functionality (or we can say hardware) is traditionally defined by the Hardware Description Language (HDL) which is not easy to 
learn. Moreover, development and debugging in such language can be time consuming. 
Due to this, higher abstractions like C/C++ or Python are being used. 
However, these abstractions typically suffer from performance issues and it isn't easy to describe high-speed network engine
in such language. On the other hand, these approaches are suitable for developers who don't have experience with HDL languages.
We feel that programming of FPGAs using the abstract description is very beneficial because user, developer, or network administrator can be
focused on network application and doesn't need to care about details of HDL language.

\section{Goals of the Dissertation Thesis}
% Tady se sepise par zakladnich prispevku moji prace (bude se pak opakovat v abstraktu, intru a zaveru)
The goals of my \thesis{} are summarized in the following list:
\begin{enumerate}
    \item To propose an architecture of network device which is suitable for automatic generation from an abstract description and is
    capable to implement high-speed packet processing devices.
    \item To identify and provide building blocks which are suitable for automatic generation and are capable 
    to work at high throughput.
    \item To introduce a transformation process for generation of a high-speed network device from the abstract description.
    \item To evaluate reached results from the view of flexibility, FPGA resources and throughput.
\end{enumerate}

\section{Structure of the Dissertation Thesis}
%The \thesis is organized into \dots chapters as follows:
%\begin{enumerate}
%\item \emph{Introduction}: Describes the motivation behind our efforts together with our goals. There is also a list of contributions of this \thesis. 
%\item \emph{Background and State-of-the-Art}: Introduces the reader to the necessary theoretical background and surveys the current state-of-the-art.
%\item \emph{Overview of Our Approach}: \dots
%\item \emph{Main Results}: \dots
%\item \emph{Conclusions}: Summarizes the results of our research, suggests possible topics for further research, and concludes the thesis.
%\end{enumerate}

The \thesis{} is organized into following chapters:
\begin{enumerate}
    \item \emph{Introduction} describes the motivation behind our efforts together with our goals. There is also a list of contributions 
    of this \thesis{}.
    \item \emph{Background and Related Work} introduces the reader to the necessary theoretical background and surveys the current 
    state-of-the-art in high-speed computer networks, together with languages for description of packet processing devices.
    \item \emph{Architecture of a Network Device} introduces possible targets, details of FPGA platform and it provides the overview of 
    our architecture of network device which is suitable for automatic generation from the abstract description. 
    The chapter also introduces our approach behind the mapping to the proposed architecture.
    \item \emph{Parser and Deparser Architecture} provides further details of input/output network blocks (parser and deparser) together with
    experimental results. 
    This chapter also discusses our transformation process from the abstract description to the architecture of these blocks.
    \item \emph{Match+Action Processing Pipeline} describes our architecture of match and action engine which is based on 
    mapping of parsed headers to executed actions. 
    The text also provides details of mapping from the abstract description to the proposed architecture of match and action engine. 
    The section also discusses the usage of current High-Level Synthesis (HLS) tool for description of processing engines at speed of 100\,Gbps.
    \item \emph{Use Case Study} introduces results for three use cases of generated pipeline, including Match+Action, to demonstrate the flexibility and
    easy extensibility with new actions and protocols.  
    This chapter provides results of required resources, throughput and possible speedup in development.
    \item \emph{Conclusion} summarizes the results of our research, suggests possible topics, and concludes the thesis.
\end{enumerate}
