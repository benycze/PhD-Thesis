\section{Motivation}
\subsection*{Contributions}
\begin{frame}
    \frametitle{Contributions}
    \begin{itemize}
        \fitem \textbf{Keywords}: FPGA, P4, Mapping to VHDL model, 100\,Gbps
        \fitem \textbf{My research is focused on mapping of abstract description to VHDL model of high-speed network device}
        \fitem I provided the following:
        %
        \begin{enumerate}
            \fitem Modular architecture of high-speed network device (\textbf{100\,Gbps} and beyond)
            \fitem Process of mapping from \textbf{P4 language} (introduced in 2013) to the architecture of high-speed network device
            \fitem Tool --- for verification of architecture and mapping process
            \fitem Overview of usage of High Level Synthesis (C/C++) in high throughput designs. 
            Results of this research were used in other research projects
        \end{enumerate}
    \end{itemize}
    \begin{block}{Contribution to the current state-of-the-art}
        I provided higher degree of flexibility to generation of FPGA based network devices from abstract description (P4 language).
    \end{block}
\end{frame}

\begin{frame}
    \frametitle{Motivation}
    \begin{itemize}
        \fitem Network devices with fixed functionality are not sufficient
        \fitem SDN (Software Defined-Networking)
        \begin{itemize}
            \fitem Promises higher degree of flexibility 
            \fitem Two components --- SDN Controller and SDN Datapath
            \fitem ($\bm{+}$) Application is implemented in controller
            \fitem ($\bm{+}$) Datapath is reconfigurable 
            \fitem ($\bm{-}$) HW devices support a limited set of protocols and actions
        \end{itemize}
        
        \fitem Current requirements on modern network devices
        \begin{itemize}
            \fitem Easy extensibility with new protocols and actions
            \fitem Capability to process data at high rates (100\,Gbps and beyond)
        \end{itemize}
        
        \fitem \textit{Field Programmable Gate Array} (FPGA) provides a reprogrammable structure $\rightarrow$ suitable as a target technology 
        \begin{itemize}
            \fitem Programmed in Hardware Description Language (HDL)
            \fitem Hard to learn
        \end{itemize}
    \end{itemize}
\end{frame}