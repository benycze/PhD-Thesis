\section{Background}
\subsection*{Fundamental Network Operations}
\begin{frame} %[allowframebreaks]
    \frametitle{Fundamental Network Operations}
    \begin{itemize}
       \fitem Fundamental network operations from the packet processing point of view
       \begin{enumerate}
           \fitem \textbf{Data Extraction} --- extraction of interesting data (i.e, protocol headers) from incoming packets
           \fitem \textbf{Classification} --- categorization of incoming packets into classes (based on extracted data)
           \fitem \textbf{Data Processing} --- perform an operation based on the assigned class (e.g., data modification, filtering, and so on)
        \end{enumerate}

        \fitem These network operations are the core of each network device
        \fitem Each operation has an influence on throughput        
    \end{itemize}
\end{frame}

\subsection*{Languages and Abstractions}
\begin{frame}[allowframebreaks]
    \frametitle{Languages and Packet Processing Abstractions}
    \begin{itemize}
        \fitem Current HLS tools - not suitable for describing high-speed network devices 
        \begin{itemize}
            \fitem The result highly depends on provided description 
            \fitem Not suitable for novices
        \end{itemize}
        %
        \fitem This drives researchers to provide domain specific languages which are suitable for computer networks 
        \fitem Such languages typically describe packet processing using the match and action model
        \begin{enumerate}
            \fitem \textbf{Gorilla}
            \begin{itemize}
                \fitem Lavasani et al. introduce the language and translation to Verilog
                \fitem Capable to hit 100\,Gbps 
                \fitem Uses templates of processing engines and packet parsers $\rightarrow$ makes the approach somewhat static 
                \fitem e.g., you have to implement the protocol parser if you want to support it
            \end{itemize}
            
            \pagebreak
            
            \fitem \textbf{SDNet}
            \begin{itemize}
                \fitem Commercial solution from Xilinx
                \fitem Introduces the PX language and translation to Verilog
                \fitem Flexible solution capable to scale from 1 to 100\,Gpbs
                \fitem Closed system $\rightarrow$ harder implementation of novel packet processing approaches 
                (not suitable for researchers)
            \end{itemize}
            
            % Put the P4 to the next slide
            \fitem \textbf{P4} (Programming Protocol-independent Packet Processors)
            \begin{itemize}
                \fitem High-level and platform-agnostic language which is developed since 2013
                \fitem Provides a way to define a packet processing functionality
                \fitem Designed to be platform independent (CPU, NPU, ASIC, FPGA)
            \end{itemize}
            
        \end{enumerate}
    \end{itemize}
\end{frame}

\subsection*{Specification of P4 Language}
\begin{frame}
    \frametitle{P4} 
    \framesubtitle{Popularized in  [P4CES16], [ROOT16]}
    \begin{itemize} 
        \fitem Relatively simple syntax\footnote{Specification of the language is available at \url{www.p4.org}}
        \fitem The language defines five basic aspects of packet processing:
        \begin{enumerate}
            \fitem \textbf{Header Format} --- defines the structure of protocol
            \fitem \textbf{Packet Parser} --- defines the process of header parsing
            \fitem \textbf{Table Specification} --- defines how extracted fields are mapped to actions
            \fitem \textbf{Action Specification} --- defines compound actions that may be executed for packets
            \fitem \textbf{Control Program} --- defines the control flow among the tables
        \end{enumerate} 
        \fitem Front end of the compiler is available under open source license 
        $\rightarrow$ compilers for different targets can be implemented      
        \fitem The next step in the SDN ecosystem $\rightarrow$ provides a way for the specification
        of SDN Datapath functionality
    \end{itemize}
\end{frame}

\subsection*{Example}
\begin{frame}[fragile,allowframebreaks]
    \frametitle{Example of P4 Program}
    \framesubtitle{Simple VLAN Tagging Device}
    % Include the file with example
    % Headers and parser % % % % % % % % % % % % % % %
\begin{minipage}[t]{0.48\textwidth}
\begin{minted}[fontsize=\scriptsize]{C}
header_type ethernet_t {
  fields {
    dAddr : 48; 
    sAddr : 48; 
    eType : 16; 
  }   
}

header_type vlan_tag_t {
  fields {
    pcp   : 3;
    cfi   : 1;
    vid   : 12; 
    eType : 16; 
  }   
}
\end{minted}
\end{minipage} 
%
\begin{minipage}[t]{0.48\textwidth}
\begin{minted}[fontsize=\scriptsize]{C}
parser start {
  return parse_ethernet;
}

header ethernet_t eth;
parser parse_ethernet {
  extract(eth);
  return select(latest.eType) {
    0x8100  : parse_vlan;
    default : ingress;
  }   
}

header vlan_tag_t vlan;
parser parse_vlan {
  extract(vlan);
  return ingress;
}   
\end{minted}
\end{minipage}

\pagebreak

% Actions, control, Table % % % % % % % % % % % % % % %
\begin{minipage}[t]{0.48\textwidth}
\begin{minted}[fontsize=\scriptsize]{C}
control ingress {
  if(valid(vlan)) {
    // We want to retag or
    // add a tag based on
    // the source MAC
    apply(retagTable);
  } else {
    // We want to add
    // the tag
    apply(tagTable);
  }
}

table retagTable {
  reads {
    vlan.vid : exact;
  }
  actions {retag; remove;}
}
\end{minted}
\end{minipage}
\begin{minipage}[t]{0.48\textwidth}
\begin{minted}[fontsize=\scriptsize]{C}
table tagTable {
  reads {
    eth.sAddr : exact;
  }
  actions {add;}
}
 
function retag(vid) {
  modify_field(vlan.vid,vid);
}

function remove() {
  modify_field(eth.eType,vlan.eType);
  remove_header(vlan);
}

function add(vid) {
  add_header(vlan);
  ...
}
    \end{minted}
\end{minipage} 
\end{frame}